% Definición clara del tipo de problema: clasificación binaria, clasificación
% multiclase o regresión.
%
% Especificación del objetivo del modelo.

\section{Formulación del problema}

El problema se formula como una tarea de \textbf{regresión} donde el objetivo es
predecir el flujo de calor del suelo \verb!soil_heat! utilizando las variables
meteorológicas medidas en diferentes niveles de la torre.

\subsection{Definición del objetivo}
El modelo busca establecer relaciones cuantitativas entre las condiciones
atmosféricas (temperatura, humedad, velocidad y dirección del viento) y el
intercambio de calor entre la superficie terrestre y la atmósfera. Esta
predicción es fundamental para:

\begin{itemize}
    \item Estudios de balance energético en ecosistemas de alta montaña
    \item Análisis de procesos de intercambio superficie-atmósfera
    \item Estimación de flujos de calor sensible y latente
    \item Modelado microclimático en regiones andinas
\end{itemize}

La variable objetivo \verb!soil_heat! representa el flujo de calor del suelo medido
a 8 cm de profundidad, expresado en W/m², donde valores positivos indican
transferencia de calor desde el suelo hacia la atmósfera y valores negativos
representan el flujo inverso.

