% Entrenamiento de al menos un modelo de aprendizaje automático.
% Presentación de resultados preliminares mediante métricas acordes al tipo de problema.

\section{Modelo entrenado y resultados preliminares}
\subsection{Modelos implementados}
Se entrenaron tres modelos de aprendizaje automático con los siguientes parámetros:

\begin{enumerate}
    \item \textbf{Regresión Lineal}: Modelo baseline con regularización estándar
    \item \textbf{Random Forest}: Ensamble de 100 árboles de decisión con paralelización
    \item \textbf{Gradient Boosting}: Boosting secuencial con 100 estimadores
\end{enumerate}

\subsection{División de datos y estrategia de validación}
\begin{itemize}
    \item \textbf{Entrenamiento}: 43,842 muestras (80\%)
    \item \textbf{Prueba}: 10,961 muestras (20\%)
    \item \textbf{Estrategia}: División aleatoria estratificada (seed=42)
    \item \textbf{Características}: 35 variables predictoras procesadas
\end{itemize}

\subsection{Métricas de evaluación}
Para evaluar el rendimiento de los modelos de regresión se utilizaron:

\begin{itemize}
    \item \textbf{R² (Coeficiente de determinación)}: Proporción de varianza explicada
    \item \textbf{MSE (Error cuadrático medio)}: Penalización cuadrática de errores grandes
    \item \textbf{MAE (Error absoluto medio)}: Medida robusta de error promedio
\end{itemize}

\subsection{Resultados experimentales}
Los modelos mostraron el siguiente rendimiento diferenciado:

\begin{table}[h]
\hspace{-1cm}
\begin{tabular}{|l|c|c|c|c|c|c|}
\hline
\textbf{Modelo} & \textbf{R² Train} & \textbf{R² Test} & \textbf{MSE Train} & \textbf{MSE Test} & \textbf{MAE Train} & \textbf{MAE Test} \\
\hline
Regresión Lineal & 0.2460 & 0.2187 & 640,539 & 656,706 & 548.18 & 545.68 \\
Random Forest & \textbf{0.9375} & \textbf{0.5588} & 53,124 & 370,880 & 112.42 & 298.30 \\
Gradient Boosting & 0.3379 & 0.3229 & 562,421 & 569,143 & 504.24 & 509.49 \\
\hline
\end{tabular}
\caption{Métricas de rendimiento comparativo de los tres modelos}
\end{table}

\subsection{Análisis de resultados}

\subsubsection{Modelo ganador: Random Forest}
\textbf{Random Forest} emergió como el modelo superior con:
\begin{itemize}
    \item \textbf{R² = 0.5588}: Explica 55.88\% de la variabilidad del flujo de calor
    \item \textbf{MSE = 370,880 W²/m$^4$}: Error cuadrático moderado
    \item \textbf{MAE = 298.30 W/m²}: Error absoluto promedio aceptable
\end{itemize}

\subsubsection{Detección de sobreajuste}
El análisis revela \textbf{sobreajuste moderado en Random Forest}:
\begin{itemize}
    \item R² Train (0.9375) $>>$ R² Test (0.5588)
    \item Diferencia de 0.38 puntos indica memorización de patrones de entrenamiento
    \item MSE aumenta significativamente de entrenamiento (53K) a prueba (371K)
\end{itemize}

\subsubsection{Rendimiento comparativo}
\begin{itemize}
    \item \textbf{Regresión Lineal}: Rendimiento consistente pero limitado (R² $\approx$ 0.22)
    \item \textbf{Gradient Boosting}: Mejor generalización que Random Forest (R² = 0.32)
    \item \textbf{Random Forest}: Mayor capacidad predictiva pero con sobreajuste
\end{itemize}

Las visualizaciones implementadas permiten evaluar:
\begin{itemize}
    \item \textbf{Predicciones vs. valores reales}: Correlación visual del modelo ganador
    \item \textbf{Análisis de residuos}: Detección de heterocedasticidad y patrones
    \item \textbf{Distribución de errores}: Verificación de normalidad en residuos
    \item \textbf{Importancia de características}: Ranking de variables meteorológicas influyentes
\end{itemize}

\subsubsection{Interpretación de resultados}
El \textbf{R² = 0.5588} indica que el modelo Random Forest captura efectivamente:
\begin{itemize}
    \item Patrones de intercambio energético superficie-atmósfera
    \item Relaciones no-lineales entre variables meteorológicas
    \item Efectos de gradientes verticales de temperatura y viento
    \item Ciclos temporales diurnos y estacionales
\end{itemize}

El error absoluto medio de \textbf{298.30 W/m²} es razonable considerando:
\begin{itemize}
    \item Rango de soil\_heat: [-4,000, +4,132] W/m² (8,132 W/m² total)
    \item Error relativo: $\sim$7.3\% del rango total
    \item Variabilidad natural alta en flujos de calor en ecosistemas montanos
\end{itemize}

