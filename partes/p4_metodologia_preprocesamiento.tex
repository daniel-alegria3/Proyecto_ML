% Análisis exploratorio de datos con apoyo de visualizaciones.
% Limpieza y preparación del dataset.
% Aplicación de técnicas básicas de ingeniería de características.

\section{Metodología de preprocesamiento}

\subsection{Análisis exploratorio de datos}
El análisis exploratorio del dataset de \textbf{54,803 registros} reveló las
siguientes características:

\begin{itemize}
    \item \textbf{Dimensiones}: 54,803 filas × 27 columnas (11.29 MB)
    \item \textbf{Período temporal}: 2018-2025 con distribución temporal continua
    \item \textbf{Variable objetivo}: soil\_heat con rango de -4,000.97 a 4,132.34 W/m²
    \item \textbf{Valores faltantes críticos}: 25.31\% en soil\_heat (13,870 registros)
\end{itemize}

\noindent\textbf{Distribución de valores faltantes}\\
El análisis identificó patrones específicos de datos faltantes:

\begin{table}[h]
\centering
\begin{tabular}{|l|c|c|}
\hline
\textbf{Variable} & \textbf{Valores Faltantes} & \textbf{Porcentaje} \\
\hline
soil\_heat & 13,870 & 25.31\% \\
temp\_n2-n6 & 1,395 & 2.55\% \\
RH\_n2-n6 & 705 & 1.29\% \\
\hline
\end{tabular}
\caption{Distribución de valores faltantes por grupo de variables}
\end{table}

Las visualizaciones implementadas incluyen análisis completo de:
\begin{itemize}
    \item Distribución estadística de la variable objetivo (soil\_heat)
    \item Series temporales multi-nivel de temperatura, viento y humedad
    \item Matriz de correlación de 27 variables meteorológicas
    \item Patrones diurnos y estacionales del flujo de calor
    \item Análisis de dispersión entre variables predictoras y objetivo
\end{itemize}

\subsection{Limpieza y preparación del dataset}
El proceso de preprocesamiento se decidio \textbf{ignorar} elementos de valores
faltantes. Ademas de realizar:

\begin{enumerate}
    \item \textbf{Construcción de índice temporal}: Timestamps precisos usando year-month-day-hour
    \item \textbf{Ordenamiento cronológico}: Organización secuencial para interpolación temporal
    \item \textbf{Interpolación temporal}: Método "time" para aprovechat continuidad temporal
    \item \textbf{Imputación residual}: SimpleImputer con estrategia de mediana para valores restantes
    \item \textbf{Normalización}: StandardScaler aplicado a 35 características finales
\end{enumerate}

Resultado del preprocesamiento: \textbf{0 valores faltantes} en todas las 27 variables.

\subsection{Ingeniería de características}
Se generaron \textbf{35 características} mediante técnicas avanzadas:

\begin{itemize}
    \item \textbf{Codificación cíclica temporal} (4 características):
    \begin{itemize}
        \item hour\_sin, hour\_cos: Captura periodicidad diaria
        \item month\_sin, month\_cos: Captura estacionalidad anual
    \end{itemize}

    \item \textbf{Gradientes verticales de temperatura} (5 características):
    \begin{itemize}
        \item temp\_gradient\_1\_2 hasta temp\_gradient\_5\_6
        \item Diferencias entre niveles consecutivos (2m-6m, 6m-12m, etc.)
    \end{itemize}

    \item \textbf{Estadísticas agregadas multi-nivel} (6 características):
    \begin{itemize}
        \item temp\_mean, temp\_std: Estadísticas de temperatura por timestamp
        \item wind\_mean, wind\_std: Estadísticas de velocidad de viento
        \item rh\_mean, rh\_std: Estadísticas de humedad relativa
    \end{itemize}

    \item \textbf{Variables originales}: 20 variables meteorológicas originales
\end{itemize}

\subsubsection{División y normalización}
\begin{itemize}
    \item \textbf{Conjunto de entrenamiento}: 43,842 muestras (80\%)
    \item \textbf{Conjunto de prueba}: 10,961 muestras (20\%)
    \item \textbf{Estrategia}: División aleatoria con seed=42
    \item \textbf{Normalización}: StandardScaler ajustado solo en entrenamiento
\end{itemize}

