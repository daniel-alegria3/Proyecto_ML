\section{Conclusiones parciales y próximos pasos}

Durante esta primera etapa del proyecto se ha completado con éxito el análisis
exploratorio, preprocesamiento y entrenamiento de modelos para abordar el
problema de regresión orientado a la predicción del flujo de calor del suelo
(\texttt{soil\_heat}) utilizando datos meteorológicos obtenidos de la Torre de
Gradiente del IGP en Junín, Perú.

El modelo con mejor desempeño lo obtuvo el modelo \textbf{Random Forest}, con un
$R^2$ en prueba de \textbf{0.5588}, una reducción significativa del
\textit{error cuadrático medio (MSE)} en comparación con los otros modelos y un
\textit{MAE} de 298.30, lo cual indica una mejor capacidad de ajuste sin
sobreajuste severo.

\subsection*{Próximos pasos}
\begin{enumerate}
    \item \textbf{Optimizar hiperparámetros} del modelo Random Forest con
        \texttt{GridSearchCV} o \texttt{RandomizedSearchCV}.
    \item \textbf{Explorar modelos adicionales} como XGBoost, LightGBM o redes
        neuronales para comparar su rendimiento.
    \item \textbf{Aplicar validación cruzada k-fold} para una evaluación más
        robusta del rendimiento.
    \item \textbf{Analizar la importancia de características} del modelo para
        identificar qué variables tienen mayor influencia sobre el flujo de
        calor del suelo.
    \item \textbf{Desarrollar visualizaciones interactivas} y/o un prototipo de
        sistema que permita visualizar predicciones en tiempo real con base en
        los datos atmosféricos.
    \item Documentar todo el \textit{pipeline} para facilitar la
        reproducibilidad del análisis.
\end{enumerate}

