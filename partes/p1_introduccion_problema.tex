% Selección de un conjunto de datos pertinente, proveniente de una fuente
% oficial de datos abiertos del Perú (https://datosabiertos.gob.pe/).

% Justificación del problema abordado, argumentando su relevancia técnica y
% social.

\section{Introducción del problema}

El monitoreo de variables meteorológicas en ecosistemas altoandinos es
fundamental para comprender los procesos físicos que ocurren en la superficie
terrestre y su interacción con la atmósfera. En este proyecto se analiza un
conjunto de datos proveniente de la Torre de Gradiente del Instituto Geofísico
del Perú (IGP), ubicada en LAMAR, Junín, a más de 3300 m.s.n.m. Esta torre
recopila información meteorológica con una alta resolución temporal (cada
minuto), incluyendo temperatura, humedad relativa, velocidad y dirección del
viento, así como flujos de calor, lo cual ofrece una visión detallada del
comportamiento climático en esta región.

El objetivo principal del proyecto es desarrollar un modelo de aprendizaje
automático de tipo regresión que permita predecir el flujo de calor del suelo
\verb!soil_heat! a partir de las variables meteorológicas registradas. El flujo de
calor del suelo es una variable clave en el balance energético de la superficie,
con implicancias directas en procesos como la evaporación, el crecimiento
vegetal, la productividad agrícola y la dinámica hídrica.

\subsection{Justificación del problema abordado}

Desde un enfoque técnico, la predicción del flujo de calor del suelo utilizando
modelos de aprendizaje automático representa un desafío interesante debido a la
naturaleza multivariada, correlacionada y altamente dinámica de los datos
atmosféricos. Comparar distintos algoritmos permite identificar cuál se ajusta
mejor al problema, en términos de precisión, eficiencia y capacidad de
generalización. Este tipo de solución basada en datos puede contribuir al
desarrollo de herramientas de pronóstico ambiental más precisas, adaptadas al
contexto geográfico y climático del Perú.

Desde una perspectiva social, la mejora en la comprensión y predicción de
variables como el flujo de calor del suelo tiene un impacto significativo en la
agricultura andina, una actividad fuertemente dependiente del clima. En regiones
como Junín, donde muchas comunidades rurales basan su subsistencia en la
agricultura familiar, disponer de modelos predictivos puede ayudar en la toma de
decisiones relacionadas con el riego, la siembra y la cosecha. Además, en un
contexto de cambio climático, contar con herramientas que permitan anticipar
comportamientos extremos o cambios en el patrón térmico del suelo es crucial
para la gestión sostenible del territorio y los recursos naturales.

Por tanto, este proyecto se sitúa en la intersección entre la ciencia de datos,
la meteorología aplicada y el desarrollo sostenible, proponiendo una solución
técnica de valor tangible para uno de los sectores clave del país.

